
\documentclass{beamer}
\usepackage{colortbl}
\usepackage{graphicx}

\usetheme{Antibes}

\title[Statistics and Data Analysis]{A Short Introduction to Statistics and Data Analysis}
\author{James T. Durant}

\begin{document}

\AtBeginSection[]
{
\begin{frame}
	\frametitle{Outline}
	\tableofcontents[currentsection]
\end{frame}
}


\begin{frame}
\titlepage
\end{frame}

\section{Introduction}
\begin{frame}{Introduction}
Today we will be demonstrating several concepts that we are utilizing in the Exposure Investigation and Data Analysis Team. We will be using \textbf{R} as a platform to demonstrate these concepts. our focus will not be on the mechnics of using \textbf{R} or the underlying mathmatics, but to try and illustrate the concepts of what is happening and basic guidelines of their use.

 \textbf{R} is not the sole platform that can perform these analyses, but the concepts are transient to all instances.

\alert{Almost all data analysis requires some level of sophistication - there are limitations and caveats to the techniques }

\end{frame}


\section{Data Visualization}


\begin{frame}{Data Visualization}
Excellent place to start!

\begin{itemize}
\item Trends
\item Distribution
\item Relationships
\item Unusual Points
\end {itemize}

\end{frame}

\begin{frame}{Conditioning with Boxplots}
Conditioning Data can Reveal Useful Information:
\includegraphics[height=2.5in]{graph1}


\end{frame}


\begin{frame}{Conditioning with Data Catagories}
\begin{center}
Conditioning Data can Reveal Useful Information:
\includegraphics[height=3in]{Figures1}
\end{center}

\end{frame}


\begin{frame}{Conditioning with Data Catagories}
\begin{center}
Conditioning Data can Reveal Useful Information:
\includegraphics[height=3in]{Figures5}

\end{center}
\end{frame}

\section{Censored Data Analysis}
\begin{frame}{Alternatives to Substitution of Methods}


\end{frame}

\subsection{Robust Regression on Order Statistics}

\begin{frame}{Robust Regression on Order Statistics - [ROS] }

\end{frame}


\subsection{Maximum Liklihood Estimation}

\begin{frame}{Maximum Liklihood Estimation - [MLE] }

\end{frame}


\subsection{Kaplan-Meier}

\begin{frame}{Kaplan Meier - [KM] }

\end{frame}

\subsection{Multiple Imputation}

\begin{frame}{Multiple Imputation - [MI] }

\end{frame}



\section{Confidence Intervals}

\subsection{Parametric Confidence Intervals}

\begin{frame}{Choice of Parametric Distribution}

\end{frame}


\subsection{Bootstrapping}

\begin{frame}{The Bootstrap}

\end{frame}
 
\begin{frame}{Limitations of Bootstraps}

\end{frame}

\subsection{Chebychev Inequalities}

\begin{frame}{Chebychev Inequality}

\end{frame}


\end{document}


